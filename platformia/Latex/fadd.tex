\def\mytitle{ASSIGNMENT-3}
\def\myauthor{KOMATINENI CHARAN}
\def\contact{chowdraycharan016@gmail.com}
\def\mymodule{Future Wireless Communications (FWC)}
\documentclass[journal, 12pt, twocolumn]{IEEEtran}
\usepackage{setspace}
\usepackage{gensymb}
\usepackage{xcolor}
\usepackage{caption}
\usepackage[hyphens,spaces,obeyspaces]{url}
\usepackage[cmex10]{amsmath}
\usepackage{mathtools}
\singlespacing
\usepackage{amsthm}
\usepackage{mathrsfs}
\usepackage{txfonts}
\usepackage{stfloats}
\usepackage{cite}
\usepackage{cases}
\usepackage{subfig}
\usepackage{longtable}
\usepackage{multirow}
\usepackage{graphicx}
\graphicspath{{./images/}}
\usepackage[colorlinks,linkcolor={black},citecolor={blue!80!black},urlcolor={blue!80!black}]{hyperref}
\usepackage[parfill]{parskip}
\usepackage{lmodern}
\usepackage{tikz}
\usepackage{circuitikz}
\usepackage{karnaugh-map}
\usepackage{pgf}
\usepackage[hyphenbreaks]{breakurl}
\usepackage{tabularx}
\usetikzlibrary{calc}
\renewcommand*\familydefault{\sfdefault}
\usepackage{watermark}
\usepackage{lipsum}
\usepackage{xcolor}
\usepackage{listings}
\usepackage{float}
\usepackage{titlesec}
\DeclareMathOperator*{\Res}{Res}
%\renewcommand{\baselinestretch}{2}
\renewcommand\thesection{\arabic{section}}
\renewcommand\thesubsection{\thesection.\arabic{subsection}}
\renewcommand\thesubsubsection{\thesubsection.\arabic{subsubsection}}
\renewcommand\thesectiondis{\arabic{section}}
\renewcommand\thesubsectiondis{\thesectiondis.\arabic{subsection}}
\renewcommand\thesubsubsectiondis{\thesubsectiondis.\arabic{subsubsection}}
\hyphenation{op-tical net-works semi-conduc-tor}
\titlespacing{\subsection}{1pt}{\parskip}{3pt}
\titlespacing{\subsubsection}{0pt}{\parskip}{-\parskip}
\titlespacing{\paragraph}{0pt}{\parskip}{\parskip}
\newcommand{\figuremacro}[5]{
    \begin{figure}[#1]
        \centering
        \includegraphics[width=#5\columnwidth]{#2}
        \caption[#3]{\textbf{#3}#4}
        \label{fig:#2}
    \end{figure}
}
\lstset{
frame=single, 
breaklines=true,
columns=fullflexible
}
\title{\mytitle}
\author{\myauthor\hspace{1em}\\\contact\\IITH\hspace{0.5em}-\hspace{0.6em}\mymodule}
\date{22-07-2023}
\def\inputGnumericTable{}                           
\lstset{
frame=single, 
breaklines=true,
columns=fullflexible
}
\begin{document}
\theoremstyle{definition}
\newtheorem{theorem}{Theorem}[section]
\newtheorem{problem}{Problem}
\newtheorem{proposition}{Proposition}[section]
\newtheorem{lemma}{Lemma}[section]
\newtheorem{corollary}[theorem]{Corollary}
\newtheorem{example}{Example}[section]
\newtheorem{definition}{Definition}[section]
\newcommand{\BEQA}{\begin{eqnarray}}
\newcommand{\EEQA}{\end{eqnarray}}
\newcommand{\define}{\stackrel{\triangle}{=}}
\bibliographystyle{article}
\vspace{3cm}
\maketitle
\tableofcontents
\pagebreak
\section{Question}

\begin{enumerate}
    \item The figure below shows the $i^{th}$ full-adder block of a binary adder circuit.$C_{i}$  is the input carry and $C_{i+1}$ is the output carry of the circuit. Assume that each logic gate has a delay of $2$ nanosecond, with no additional time delay due to the interconnecting wires. Of the inputs $A_{i}$ , $B_{i}$ are available and stable throughout the carry propagation, the maximum time taken for an input $C_{i}$ to produce a steady-state output $C_{i+1}$ is $\underline{\hspace{2cm}}$ nanosecond.
\end{enumerate}


\begin{figure}[h]
	\centering
        \begin{circuitikz}
    \draw (1.5, 1.5) node[xor port] (xor1) {};
    \draw (xor1.in 1) -- ++(-1.0,0) node[anchor=east] {$A_i$};
    \draw (xor1.in 2) -- ++(-1.0,0) node[anchor=east] {$B_i$};
    \draw (1.5, 1.5) -- (3.0, 1.5);
    \draw (-0.1, 1.3) -- (-0.1, -1.2);
    \draw (-0.5, 1.8 ) -- (-0.5, -1.8);
    \draw (-0.1, -1.2) -- (0.5, -1.2);
    \draw (-0.5, -1.8) -- (0.5, -1.8);
    \draw (1.8, -1.5) node[and port] (and 1) {};
    \draw (4.1, 1.2) node[xor port] (xor2)  {};
    \draw (2.8, 0.9) -- (2.8, -3.3);
    \draw (2.8, -3.3) -- (-1.0, -3.3);
    \draw (-1.0,-3.3) node[anchor=east] {$C_in$};
    \draw (2.0, 1.5) -- (2.0, -0.3);
    \draw (2.0, -0.3) -- (3.1, -0.3);
    \draw (4.2, -0.6) node[and port] (and 2) {};
    \draw (1.8, -1.5) -- (5.8, -1.5);
    \draw (4.2, -0.6) -- (6.0, -0.6);
    \draw (7.2, -0.9) node[xor port] (xor3) {};
    \draw (5.8, -1.5) -- (5.8, -1.2);
    \draw (4.8, 1.2) node[anchor = east] {$S_i$};
    \draw (8.7, -0.9) node[anchor=east] {$C_i+1$};
\end{circuitikz}

        \caption{full adder}
        \label{figs:fulladder.}
\end{figure}

\pagebreak
\section{Components}
\begin{table}[h]
  \centering
   \begin{tabular}{|c|c|c|}
   \hline
   \textbf{Component}& \textbf{Values} & \textbf{Quantity}\\
\hline
ArduinoUNO &  & 1 \\  
\hline
JumperWires& M-M & 5 \\ 
\hline
Breadboard &  & 1 \\
\hline
LED & &5 \\
\hline
Resistor & 220ohms & 5\\
\hline
   \end{tabular}
   \end{table}
\begin{center}
    figure.a
\end{center}
\section{TruthTable}
\begin{table}[h]
    \centering
    \begin{tabular}{|c|c|c|c|c|}
    \hline
    \textbf{$A$} & \textbf{$B$} & \textbf{$C_{in}$} & \textbf{$S$} & \textbf{$C_{out}$ }\\
\hline
 0 & 0 & 0 & 0 & 0\\  
\hline
 0 & 0 & 1 & 1 & 0\\
\hline
0 & 1 & 0 & 1 & 0\\ 
\hline
0 & 1 & 1 & 0 & 1\\ 
\hline
 1 & 0 & 0 & 1 & 0\\ 
\hline
 1 & 0 & 1 & 0 & 1\\
\hline
1 & 1 & 0 & 0 & 1\\
\hline
1 & 1 & 1 & 1 & 1\\ 
\hline
     \end{tabular}
     \caption{Truth Table for "full adder" }
	\label{table:1}
\end{table}
\pagebreak
\section{K-Maps}
\centering
\begin{karnaugh-map}[4][2][1][$BC_{in}$][$A$]
 \minterms{1,2,4,7}
\end{karnaugh-map}
 \begin{center}
    K-Map for sum equation.
\end{center}
\begin{align}
  S = A\oplus B\oplus C_{in}
\end{align}
\centering
    \begin{karnaugh-map}[4][2][1][$BC_{in}$][$A$]
 \minterms{3,5,6,7}
 \implicant{3}{7}
\implicant{7}{6}
\implicant{5}{7}
\end{karnaugh-map}
\begin{center}
    K-Map for carry equation.
\end{center}
\begin{align}
    C_{out} = AB+C_{in}(A \oplus B)
\end{align}
\pagebreak
\section{Implimentation}
\begin{table}[h]
    \centering
    \begin{tabular}{|c|c|c|c|}
    \hline
    \textbf{Arduino Pin} & \textbf{resistor} & \textbf{input} & \textbf{output} \\
\hline
5 & resistor & $A$ &   \\ 
\hline
4 & resistor & $B$ & \\
\hline
3 & resistor & $C_{in}$ & \\
\hline
8 & resistor &  & $S$ \\
\hline
9 & resistor & & $C_{out}$\\
\hline
     \end{tabular}
     \caption{implementation}
	\label{table:2}
\end{table}
\section{Procedure}
\begin{enumerate}
    \item   Connect the circuit as per the above table.
 \item The leds 1, 2 and 3 represent the values of  inputs $A$, $B$  and $C_{in}$ respectively.
 \item The leds 4 and 5 represent the output values i.e $S$ and $C_{out}$ respectively.
 \item  Execute the circuits using the below code.
 \begin{table}[h]
\centering
	\begin{tabular}{|c|}
	\hline
	https://github.com/komatinenicharan/FWC/blob/main/platformia/code/ch.cpp\\
	\hline
\end{tabular}
\end{table}
\end{enumerate}

\end{document}mentclass{article}
\usepackage{setspace}
\usepackage{gensymb}
\usepackage{xcolor}
\usepackage{caption}
\usepackage[hyphens,spaces,obeyspaces]{url}
\usepackage[cmex10]{amsmath}
\usepackage{mathtools}
\singlespacing
\usepackage{amsthm}
\usepackage{mathrsfs}
\usepackage{txfonts}
\usepackage{stfloats}
\usepackage{cite}
\usepackage{cases}
\usepackage{subfig}
\usepackage{longtable}
\usepackage{multirow}
\twocolumn
\usepackage{graphicx}
\graphicspath{{./images/}}
\usepackage[colorlinks,linkcolor={black},citecolor={blue!80!black},urlcolor={blue!80!black}]{hyperref}
\usepackage[parfill]{parskip}
\usepackage{lmodern}
\usepackage{tikz}
\usepackage{circuitikz}
\usepackage{karnaugh-map}
\usepackage{pgf}
\usepackage[hyphenbreaks]{breakurl}
\usepackage{tabularx}
\usetikzlibrary{calc}
\renewcommand*\familydefault{\sfdefault}
\usepackage{watermark}
\usepackage{lipsum}
\usepackage{xcolor}
\usepackage{listings}
\usepackage{float}
\usepackage{titlesec}
\DeclareMathOperator*{\Res}{Res}
%\renewcommand{\baselinestretch}{2}
\renewcommand\thesection{\arabic{section}}
\renewcommand\thesubsection{\thesection.\arabic{subsection}}
\renewcommand\thesubsubsection{\thesubsection.\arabic{subsubsection}}
\renewcommand\thesectiondis{\arabic{section}}
\renewcommand\thesubsectiondis{\thesectiondis.\arabic{subsection}}
\renewcommand\thesubsubsectiondis{\thesubsectiondis.\arabic{subsubsection}}
\hyphenation{op-tical net-works semi-conduc-tor}
\titlespacing{\subsection}{1pt}{\parskip}{3pt}
\titlespacing{\subsubsection}{0pt}{\parskip}{-\parskip}
\titlespacing{\paragraph}{0pt}{\parskip}{\parskip}
\newcommand{\figuremacro}[5]{
    \begin{figure}[#1]
        \centering
        \includegraphics[width=#5\columnwidth]{#2}
        \caption[#3]{\textbf{#3}#4}
        \label{fig:#2}
    \end{figure}
}
\lstset{
frame=single, 
breaklines=true,
columns=fullflexible
}
\title{\mytitle}
\author{\myauthor\hspace{1em}\\\contact\\IITH\hspace{0.5em}-\hspace{0.6em}\mymodule}
\date{23-03-2023}
\def\inputGnumericTable{}                           
\lstset{
frame=single, 
breaklines=true,
columns=fullflexible
}
\begin{document}
\theoremstyle{definition}
\newtheorem{theorem}{Theorem}[section]
\newtheorem{problem}{Problem}
\newtheorem{proposition}{Proposition}[section]
\newtheorem{lemma}{Lemma}[section]
\newtheorem{corollary}[theorem]{Corollary}
\newtheorem{example}{Example}[section]
\newtheorem{definition}{Definition}[section]
\newcommand{\BEQA}{\begin{eqnarray}}
\newcommand{\EEQA}{\end{eqnarray}}
\newcommand{\define}{\stackrel{\triangle}{=}}
\bibliographystyle{article}
\vspace{3cm}
\maketitle
\tableofcontents
\pagebreak
\section{Question}

\begin{enumerate}
    \item The figure below shows the $i^{th}$ full-adder block of a binary adder circuit.$C_{i}$  is the input carry and $C_{i+1}$ is the output carry of the circuit. Assume that each logic gate has a delay of $2$ nanosecond, with no additional time delay due to the interconnecting wires. Of the inputs $A_{i}$ , $B_{i}$ are available and stable throughout the carry propagation, the maximum time taken for an input $C_{i}$ to produce a steady-state output $C_{i+1}$ is $\underline{\hspace{2cm}}$ nanosecond.
\end{enumerate}


\begin{figure}[h]
	\centering
        \begin{circuitikz}
    \draw (1.5, 1.5) node[xor port] (xor1) {};
    \draw (xor1.in 1) -- ++(-1.0,0) node[anchor=east] {$A_i$};
    \draw (xor1.in 2) -- ++(-1.0,0) node[anchor=east] {$B_i$};
    \draw (1.5, 1.5) -- (3.0, 1.5);
    \draw (-0.1, 1.3) -- (-0.1, -1.2);
    \draw (-0.5, 1.8 ) -- (-0.5, -1.8);
    \draw (-0.1, -1.2) -- (0.5, -1.2);
    \draw (-0.5, -1.8) -- (0.5, -1.8);
    \draw (1.8, -1.5) node[and port] (and 1) {};
    \draw (4.1, 1.2) node[xor port] (xor2)  {};
    \draw (2.8, 0.9) -- (2.8, -3.3);
    \draw (2.8, -3.3) -- (-1.0, -3.3);
    \draw (-1.0,-3.3) node[anchor=east] {$C_in$};
    \draw (2.0, 1.5) -- (2.0, -0.3);
    \draw (2.0, -0.3) -- (3.1, -0.3);
    \draw (4.2, -0.6) node[and port] (and 2) {};
    \draw (1.8, -1.5) -- (5.8, -1.5);
    \draw (4.2, -0.6) -- (6.0, -0.6);
    \draw (7.2, -0.9) node[xor port] (xor3) {};
    \draw (5.8, -1.5) -- (5.8, -1.2);
    \draw (4.8, 1.2) node[anchor = east] {$S_i$};
    \draw (8.7, -0.9) node[anchor=east] {$C_i+1$};
\end{circuitikz}

        \caption{full adder}
        \label{figs:fulladder.}
\end{figure}

\pagebreak
\section{Components}
\begin{table}[h]
  \centering
   \begin{tabular}{|c|c|c|}
   \hline
   \textbf{Component}& \textbf{Values} & \textbf{Quantity}\\
\hline
ArduinoUNO &  & 1 \\  
\hline
JumperWires& M-M & 5 \\ 
\hline
Breadboard &  & 1 \\
\hline
LED & &5 \\
\hline
Resistor & 220ohms & 5\\
\hline
   \end{tabular}
   \end{table}
\begin{center}
    figure.a
\end{center}
\section{TruthTable}
\begin{table}[h]
    \centering
    \begin{tabular}{|c|c|c|c|c|}
    \hline
    \textbf{$A$} & \textbf{$B$} & \textbf{$C_{in}$} & \textbf{$S$} & \textbf{$C_{out}$ }\\
\hline
 0 & 0 & 0 & 0 & 0\\  
\hline
 0 & 0 & 1 & 1 & 0\\
\hline
0 & 1 & 0 & 1 & 0\\ 
\hline
0 & 1 & 1 & 0 & 1\\ 
\hline
 1 & 0 & 0 & 1 & 0\\ 
\hline
 1 & 0 & 1 & 0 & 1\\
\hline
1 & 1 & 0 & 0 & 1\\
\hline
1 & 1 & 1 & 1 & 1\\ 
\hline
     \end{tabular}
     \caption{Truth Table for "full adder" }
	\label{table:1}
\end{table}
\pagebreak
\section{K-Maps}
\centering
\begin{karnaugh-map}[4][2][1][$BC_{in}$][$A$]
 \minterms{1,2,4,7}
\end{karnaugh-map}
 \begin{center}
    K-Map for sum equation.
\end{center}
\begin{align}
  S = A\oplus B\oplus C_{in}
\end{align}
\centering
    \begin{karnaugh-map}[4][2][1][$BC_{in}$][$A$]
 \minterms{3,5,6,7}
 \implicant{3}{7}
\implicant{7}{6}
\implicant{5}{7}
\end{karnaugh-map}
\begin{center}
    K-Map for carry equation.
\end{center}
\begin{align}
    C_{out} = AB+C_{in}(A \oplus B)
\end{align}
\pagebreak
\section{Implimentation}
\begin{table}[h]
    \centering
    \begin{tabular}{|c|c|c|c|}
    \hline
    \textbf{Arduino Pin} & \textbf{resistor} & \textbf{input} & \textbf{output} \\
\hline
5 & resistor & $A$ &   \\ 
\hline
4 & resistor & $B$ & \\
\hline
3 & resistor & $C_{in}$ & \\
\hline
8 & resistor &  & $S$ \\
\hline
9 & resistor & & $C_{out}$\\
\hline
     \end{tabular}
     \caption{implementation}
	\label{table:2}
\end{table}
\section{Procedure}
\begin{enumerate}
    \item   Connect the circuit as per the above table.
 \item The leds 1, 2 and 3 represent the values of  inputs $A$, $B$  and $C_{in}$ respectively.
 \item The leds 4 and 5 represent the output values i.e $S$ and $C_{out}$ respectively.
 \item  Execute the circuits using the below code.
 \begin{table}[h]
\centering
	\begin{tabular}{|c|}
	\hline
	https://github.com/komatinenicharan/FWC/blob/main/platformia/code/ch.cpp\\
	\hline
\end{tabular}
\end{table}
\end{enumerate}

\end{document}
